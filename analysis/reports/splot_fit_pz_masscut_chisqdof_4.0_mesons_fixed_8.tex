
\begin{table}
    \begin{center}
        \begin{tabular}{lr}\toprule
            Parameter & Value \\\midrule
            Signal Yield & $(1.68557 \pm 0.00010) \times 10^{4}$ \\
            Background Yield $\#1$ & $(1.68557 \pm 0.00010) \times 10^{4}$ \\
            Background Yield $\#2$ & $(1.68557 \pm 0.00010) \times 10^{4}$ \\
            Background Yield $\#3$ & $(1.68557 \pm 0.00010) \times 10^{4}$ \\
            Background Yield $\#4$ & $(1.68557 \pm 0.00010) \times 10^{4}$ \\
            Background Yield $\#5$ & $(1.68557 \pm 0.00010) \times 10^{4}$ \\
            Background Yield $\#6$ & $(1.68557 \pm 0.00010) \times 10^{4}$ \\
            Background Yield $\#7$ & $(1.68557 \pm 0.00010) \times 10^{4}$ \\
            Background Yield $\#8$ & $(1.68557 \pm 0.00010) \times 10^{4}$ \\\bottomrule
        \end{tabular}
        \caption{The parameter values and uncertainties for the sPlot fit of data with $\chi^2_\nu < 4.00$ using 8 fixed background slope(s). Uncertainties are calculated using the covariance matrix of the fit. All $\lambda$ parameters have units of $\si{\nano\second}^{-1}$.}
    \end{center}
\end{table}
% 16855.297112911772 weighted events